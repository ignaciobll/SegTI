\documentclass[11pt,a4paper]{article}

\usepackage[utf8]{inputenc}
\usepackage[spanish]{babel}

\usepackage{verbatim}

\title{Análisis del \emph{malware MiniDuke}}

\author{\begin{tabular}[center]{c}
          Ignacio Ballesteros González \\
          \small w140062 \\
          \small 05448027V \\
        \end{tabular}
      }

\date{\today}

\begin{document}

\maketitle
\tableofcontents
\normalsize
\begin{abstract}
  Segunda práctica de la asignatura de \emph{Seguridad de las
    Tecnologías de la Información}. Se realizará el estudio de una
  muestra del código malicioso \emph{MiniDuke}. Se ha elegido la
  opción \textit{(c)} de estudio en base a las normas establecidas.

  \begin{center}
    $1 + ((5448027 * 726391632)~mod~330) = 175$
  \end{center}
\end{abstract}

\section{Introducción}
\label{sec:intro}

\paragraph{Código malicioso}

\paragraph{Tipo}

\paragraph{Familia}

\section{Ficha resumen}
\label{sec:resumen}

\paragraph{Denominación}

\paragraph{Origen/autor}

\paragraph{Destinatario}

\paragraph{Fecha de lazamiento}

\paragraph{Fecha de descubrimiento}

\paragraph{Tipo de código malicioso}

\paragraph{Funcionamiento general}

\subparagraph{Modo de Infección}

\subparagraph{Modo de replicación}

\subparagraph{Modo de propagación}

\subparagraph{Modo de ocultación}

\subparagraph{Ejecución de la carga}

\paragraph{Tiempo de vulnerabilidad relacionada}

\paragraph{Modo de desinfección}

\paragraph{Ejemplo de ataque donde se ha empleado}

\paragraph{Medidas de seguridad tomadas tras su descubrimiento}

\paragraph{Resto de miembros de su familia}

\paragraph{Otra información relevante}

\section{Conclusiones}
\label{sec:conclusiones}


\end{document}
