% \documentclass[11pt,twocolumn]{article}
\documentclass[11pt,a4paper]{article}

\usepackage[utf8]{inputenc}
\usepackage[spanish]{babel}

\usepackage{blindtext}
\usepackage{scrextend}
\usepackage{enumitem}

\usepackage{graphicx}
\usepackage{pgfplots}
\usepackage{tikz}
\usepackage{tikzscale}

\usepackage[subpreambles=true]{standalone}

\usepackage{subfigure}

\usepackage{caption}

\title{Análisis del efecto Avalancha sobre XTEA}

\author{Ignacio Ballesteros González}

\date{\today}

\begin{document}
\maketitle
\tableofcontents
\begin{abstract}
  Para la realización de la segunda práctica de la asignatura de
  \emph{Seguridad de las Tecnologías de la Información} se ha
  realizado el análisis del efecto \emph{avalancha} sobre el algoritmo de
  cifrado \emph{XTEA}.
\end{abstract}

\section{Introducción}

El algoritmo \texttt{XTEA} fue desarrollado...

\subsection{Herramientas utilizadas}

Para la posible replicación de este análisis, se detallan las
herramientas utilizadas en el desarrollo del mismo:

\begin{description}[align=left] %, noitemsep]
\item [Python 2 y 3] Plataforma de desarrollo.
\item [Emacs] Entorno de desarrollo para programación en \emph{Python}. \emph{Emacs 25.3.2}
\item [xtea] Implementación del algoritmo de cifrado \emph{xtea}. \emph{xtea (0.4.0)}
\item [scipy] Librería \emph{Python} para cálculos estadísticos.
\item [\LaTeX] Realización de figuras y de este mismo documento.
\end{description}

\section{Estructura del análisis}

Para la realización del análisis se ha seguido el siguiente
procedimiento:

\begin{enumerate}
\item Generamos una muestra de estudio (\emph{pseudo-aleatoria}) compuesta de una clave
  \textbf{(K)}, un vector de inicialización \textbf{(VI)} y un texto a
  cifrar \textbf{(T)}.
\item Realizamos un cifrado con \emph{XTEA} para obtener la cadena
  cifrada \textbf{C}.
\item Para comprobar los distintos efectos avalancha modificamos un
  bit en la clave, el vector inicial y el texto \textbf{(K', VI',
    T')}.
\item Para cada modificación, se aplica de nuevo el cifrado con las
  combinaciones de clave, vector inicial y texto:
  \begin{itemize}
  \item (K, VI, T')
  \item (K, VI', T)   
  \item (K', VI, T)
  \end{itemize}
\item Por cada resultado del cifrado, se calcula la distancia de
  Hamming con respecto a la cadena cifrada original \textbf{C}.
\item Aplicamos un análisis estadístico sobre los histogramas generados.
\end{enumerate}

\section{Datos recogidos}

Los análisis se ha realizado según el tamaño de la
muestra. Los datos estadísticos asociados se presentan en la siguiente
tabla:

\begin{center}
  \vspace{-15pt}
  \label{tab:medidas}
  \begin{tabular}[c]{| r | *{6}{c} |}
  \hline
  Población  & Media  & Moda & Mediana & Desviación & Asimetría & Curtosis \\ \hline
  10         & 158.36 & 157  & 158     & 7.43       & -0.82     & 3.05     \\
  100        & 159.93 & 165  & 161     & 8.68       & 0.29      & 0.85     \\
  1000       & 160.64 & 165  & 161     & 9.10       & -0.11     & -0.23    \\
  10000      & 160.29 & 161  & 160     & 8.99       & 0.02      & 0.04     \\
  100000     & 160.23 & 161  & 160     & 8.97       & 0.00      & 0.00     \\
  1000000    & 160.24 & 161  & 160     & 8.97       & 0.00      & -0.01    \\
  100000000  & 160.03 & 160  & 160     & 8.97       & 0.00      & 0.00     \\
  \hline
  \end{tabular}
  \captionof{table}{Tabla de medidas estadísticas}

\vspace{-10pt}
\end{center}


\begin{figure}[h]
  \centering
  \includestandalone[mode=buildnew]{./img/test}
  \caption{Análisis sobre una población de 1000}
\end{figure}


\section{Valoración de los resultados}

Cuando se trata de comprobar el efecto \emph{avalancha} sobre una
función de cifrado se espera que cuando se cambie una pequeña parte
de la cadena de entrada (por ejemplo, un \emph{bit}), aproximadamente
la mitad de los \emph{bits} de la cadena de salida cambien también.

Que solo cambie aproximadamente la mitad de los bits quiere decir que
la \emph{distancia de Hamming} asociada a la modificación de un solo
\emph{bit} sea de aproximadamente la \textbf{mitad} de la cadena. En
en el caso de la función \emph{XTEA} las cadenas son de \emph{128
  bits}, por lo que la \emph{distancia de Hamming} asociada en un
cambio de \emph{bit} se concentraría cerca de $64$. Este análisis se
ha hecho con esta longitud de cadena porque el cifrado toma bloques de
\emph{128 bits}.


\end{document}
