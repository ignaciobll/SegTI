%% \documentclass[11pt,twocolumn]{article}
\documentclass[11pt,a4paper]{article}

\usepackage[utf8]{inputenc}
\usepackage[spanish]{babel}

\usepackage{graphicx}
\usepackage{pgfplots}
\usepackage{tikz}
\usepackage{tikzscale}

\usepackage{subfigure}

\usepackage{caption}

\title{Análisis del efecto Avalancha sobre XTEA}

\author{Ignacio Ballesteros González}

\date{\today}

\begin{document}
\maketitle
\tableofcontents
\begin{abstract}
  Para la realización de la segunda práctica de la asignatura de
  \emph{Seguridad de las Tecnologías de la Información} se ha
  realizado el análisis del efecto \emph{avalancha} sobre el algoritmo de
  cifrado \emph{XTEA}.
\end{abstract}

\section{Herramientas utilizadas}

Para la posible replicación de este análisis, se detallan las
herramientas utilizadas en el desarrollo del mismo:

\begin{description}[align=left] %, noitemsep]
\item [Python 2 y 3] Plataforma de desarrollo.
\item [Emacs] Entorno de desarrollo para programación en \emph{Python}. \emph{Emacs 25.3.2}
\item [xtea] Implementación del algoritmo de cifrado \emph{xtea}. \emph{xtea (0.4.0)}
\item [as] Librería \emph{Python} para cálculos estadísticos.
\item [\LaTeX] Realización de figuras y de este mismo documento.
\end{description}


\end{document}
