% \documentclass[11pt,twocolumn]{article}
\documentclass[11pt,a4paper]{article}

\usepackage[a4paper]{geometry}

\usepackage[utf8]{inputenc}
\usepackage[spanish]{babel}

\usepackage{blindtext}
\usepackage{scrextend}
\usepackage{enumitem}

\usepackage{graphicx}
\usepackage{pgfplots}
\usepackage{tikz}
\usepackage{tikzscale}

\usepackage[subpreambles=true]{standalone}

\usepackage{subfigure}

\usepackage{float}

\usepackage{caption}

\title{Análisis del efecto Avalancha sobre XTEA}

\author{Ignacio Ballesteros González}

\date{\today}

\begin{document}
\maketitle
\tableofcontents
\begin{abstract}
  Para la realización de la segunda práctica de la asignatura de
  \emph{Seguridad de las Tecnologías de la Información} se ha
  realizado el análisis del efecto \emph{avalancha} sobre el algoritmo de
  cifrado \emph{XTEA}.
\end{abstract}

\section{Introducción}

El algoritmo \texttt{XTEA} fue desarrollado...

\subsection{Herramientas utilizadas}

Para la posible replicación de este análisis, se detallan las
herramientas utilizadas en el desarrollo del mismo:

\begin{description}[align=left] %, noitemsep]
\item [Python 2 y 3] Plataforma de desarrollo.
\item [Emacs] Entorno de desarrollo para programación en \emph{Python}. \emph{Emacs 25.3.2}
\item [xtea] Implementación del algoritmo de cifrado \emph{xtea}. \emph{xtea (0.4.0)}
\item [scipy] Librería \emph{Python} para cálculos estadísticos.
\item [\LaTeX] Realización de figuras y de este mismo documento.
\end{description}

\section{Estructura del análisis}

Para la realización del análisis se ha seguido el siguiente
procedimiento:

\begin{enumerate}
\item Generamos una muestra de estudio (\emph{pseudo-aleatoria}) compuesta de una clave
  \textbf{(K)}, un vector de inicialización \textbf{(VI)} y un texto a
  cifrar \textbf{(T)}.
\item Realizamos un cifrado con \emph{XTEA} para obtener la cadena
  cifrada \textbf{C}.
\item Para comprobar los distintos efectos avalancha modificamos un
  bit en la clave, el vector inicial y el texto \textbf{(K', VI',
    T')}.
\item Para cada modificación, se aplica de nuevo el cifrado con las
  combinaciones de clave, vector inicial y texto:
  \begin{itemize}
  \item (K, VI, T')
  \item (K, VI', T)   
  \item (K', VI, T)
  \end{itemize}
\item Por cada resultado del cifrado, se calcula la distancia de
  Hamming con respecto a la cadena cifrada original \textbf{C}.
\item Aplicamos un análisis estadístico sobre los histogramas generados.
\end{enumerate}

\section{Datos recogidos}

El cifrado \emph{XTEA} trabaja con bloques de $128$ \emph{bits}. Por
este motivo, hemos elegido una generación de cadenas de entrada de 128
\emph{bits} (16 \emph{bytes}) para estudiar el efecto avalancha.

Durante sucesivas iteraciones con distintos tamaños de muestra, cuando
nos acercamos a las 500 cadenas de muestra, el error de la varianza
desciende por debajo del $5\%$.

Dado que el cifrado tiene 3 parámetros de entrada (clave, vector
inicial, texto a cifrar), el estudio del efecto avalancha lo tendremos
que hacer sobre variaciones en estos 3 parámetros.

\subsection{Variación del texto de entrada}

Los datos recogidos variando en un \emph{bit} el texto a cifrar (y manteniendo clave y vector inicial) son los siguientes:

\begin{center}
  \label{tab:text}
  \begin{tabular}[c]{| r | *{6}{c} |}
  \hline
    Muestra & Media & Mediana & Desviación & Variance & Asimetría & Curtosis \\ \hline
    500.00  & 1.00  &  1.00   & 0	   &   0      &  0	  & -3.00    \\
  \hline
  \end{tabular}
  \captionof{table}{Tabla de medidas estadísticas}
\end{center}

Es muy llamativo cómo nos encontramos con unos datos muy distintos a
los posibles esperados cuando se produce un efecto avalancha. Que no haya desviación, ni varianza y la distribución sea simétrica, implica que todos los valores se encuentran en 1. Es decir, cuando se modifica un bit en la entrada, solo se modifica un bit en la salida. En la figura 1 se puede observar cómo la distribución del histograma dista mucho de ser binomial. De hecho, en todos los casos comprobados, la variación es un único \emph{bit}.

\begin{figure}[H]
  \centering
  \label{fig:text}
  \includestandalone[mode=buildnew]{./img/text500}
  \caption{Análisis sobre una población de 500}
\end{figure}

\subsection{Variación en la clave}

Con una variación de un \textbf{bit} en la clave de cifrado y manteniendo el texto de entrada original obtenemos los siguientes datos:

\begin{center}
  \label{tab:key}
  \begin{tabular}[c]{| r | *{6}{c} |}
  \hline
    Muestra & Media & Mediana & Desviación & Variance & Asimetría & Curtosis \\ \hline
    500.00  & 64.27 &   64.00 &       5.55 &    30.83 &      0.08 &     0.15 \\
  \hline
  \end{tabular}
  \captionof{table}{Tabla de medidas estadísticas}
\end{center}

Y el con el histograma en la figura 2.

\begin{figure}[H]
  \centering
  \label{fig:key}
  \includestandalone[mode=buildnew]{./img/key500}
  \caption{Análisis sobre una población de 500}
\end{figure}

Si nos fijamos en los errores obtenidos respecto a una distribución
binomial $(p = 128, n = 0.5)$, el error en la varianza es del 0,036 $(3'6\%)$


\subsection{Variación en el vector inicial}

Con una variación de un \textbf{bit} en el vector de inicialización y manteniendo el texto de entrada original obtenemos los siguientes datos:

+\begin{center}
  \label{tab:vi}
  \begin{tabular}[c]{| r | *{6}{c} |}
  \hline
    Muestra & Media & Mediana & Desviación & Variance & Asimetría & Curtosis \\ \hline
    500.00 &  64.31 &   64.00 &       5.66 &    32.04 &     -0.09 &     0.34 \\
  \hline
  \end{tabular}
  \captionof{table}{Tabla de medidas estadísticas}
\end{center}

Y el con el histograma en la figura 2.

\begin{figure}[H]
  \centering
  \label{fig:vi}
  \includestandalone[mode=buildnew]{./img/vi500}
  \caption{Análisis sobre una población de 500}
\end{figure}

En el caso del vector inicial, el error en la varianza obtenido es de
$0.1\%$. Una aproximación muy buena para la binomial.

\section{Valoración de los resultados}

Cuando se trata de comprobar el efecto \emph{avalancha} sobre una
función de cifrado se espera que cuando se cambie una pequeña parte
de la cadena de entrada (por ejemplo, un \emph{bit}), aproximadamente
la mitad de los \emph{bits} de la cadena de salida cambien también.

Que solo cambie aproximadamente la mitad de los bits quiere decir que
la \emph{distancia de Hamming} asociada a la modificación de un solo
\emph{bit} sea de aproximadamente la \textbf{mitad} de la cadena. En
en el caso de la función \emph{XTEA} las cadenas son de \emph{128
  bits}, por lo que la \emph{distancia de Hamming} asociada en un
cambio de \emph{bit} se concentraría cerca de $64$. Este análisis se
ha hecho con esta longitud de cadena porque el cifrado toma bloques de
\emph{128 bits}.

El resultado arrojado por la variación de un \emph{bit} en la cadena
de texto de entrada provoca que solo se cambie un \emph{bit} en la
cadena de salida. Esto indica que no hay un buen efecto avalancha en
el cifrado.

Sin embargo, si nos centramos en el cambio de un \emph{bit} en la
clave o en el vector inicial, observamos que el efecto avalancha se
produce satisfactoriamente.

Con estos resultados, podemos decir que el cifrado \emph{XTEA} posee
parcialmente la propiedad del efecto avalancha.

\end{document}
