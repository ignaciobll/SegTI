\documentclass[11pt,a4paper]{article}

\usepackage[a4paper]{geometry}
\usepackage[utf8]{inputenc}
\usepackage[spanish]{babel}

\title{
  ISO/IEC 21827\\
  Systems Security Engineering — Capability Maturity Model® (SSE-CMM®)
}

\author{\begin{tabular}[center]{c}
          Ignacio Ballesteros González \\
          \small w140062 \\
          \small 05448027V \\
        \end{tabular}
      }
\date{\today}


\begin{document}
\maketitle

A causa del interés de clientes y proveedores en crear un marco que
proporcione herramientas de mejora en las aplicaciones y principios de
la ingeniería de la seguridad, \emph{ISO}\footnote{International
  Organization for Standarization} e \emph{IEC}\footnote{International
  Electrotechnical Commision} se desarrolla el estándar
\emph{SSE-CMM®}.

\section{Introducción}q
\label{sec:introduccion}

El estándar \emph{SSE-CMM®} busca crear una herramienta para las
organizaciones de ingeniería de la seguridad que sirva para medir sus-
prácticas, definir mejoras, establecer un indicador de confianza en la
organización y evaluar la capacidad en seguridad de un proveedor.

Los fines de \emph{SSE-CMM®} es proporcionar un marco en el que los
productos y procesos desarrollados tengan la capacidad de ser
continuados y reproducibles. Además de buscar la eficiencia y
seguridad de los mismos. La \emph{SSE-CMM®} no pretende sustituir los
procesos de las organizaciones, sino que estos se enmarquen en ella.

La realización de \emph{SSE-CMM®} ha contado con la participación de
más de 50 organizaciones y con grupos de trabajo pequeños, medianos y
grandes; además de contar con una evaluación y comentarios de críticos
independientes.

\section{Arquitectura Modelo}
\label{sec:arquitectura}

La \emph{SSE-CMM®} divide la ingeniería de seguridad en tres areas
principales: riesgo, ingeniería y garantía. Estas areas no son
independientes entre sí.

\begin{description}
\item[Riesgo] Combinación de amenazas, vulnerabilidades e impacto.
\item[Ingeniería] Proceso de creación de soluciones enmarcadas en un
  marco de restricciones y requisitos.
\item[Garantía] Grado de confianza que satisface según \emph{NIST94a}.
\end{description}

\subsection{Modelo básico}
\label{sec:basico}

La \emph{SSE-CMM®} tiene dos dimensiones, \emph{dominio} y
\emph{capacidad}. El \emph{dominio} se compone de todas las prácticas
que colectivamente definen la ingeniería de la seguridad. La
\emph{capacidad} representa prácticas de los modelos de gestión.

Estas dos dimensiones se distribuyen en un modelo matricial que
permite ubicar la capacidad técnica de ingeniería de la seguridad de
una organización.

La \emph{SSE-CMM®} enumera una serie de prácticas y procesos que se
relacionan con el \emph{ciclo de vida} de un proceso.

\subsection{Prácticas básicas}
\label{sec:practicas}

La \emph{SSE-CMM®} enumera y describe las prácticas básicas que hay en
un proceso de seguridad. Sin embargo, no especifica que toda
organización deba poseer o evaluar todas ellas, ya que puede que tenga
una sola o una mezcla de varias.

La \emph{SSE-CMM®} menciona 11 prácticas básicas, ampliadas
posteriormente en un anexo.

\begin{description}
\item[Administración de controles de seguridad] Asegurarse de que se
  cumplen los procesos de control de seguridad. Se contempla la
  asignación de responsabilidades, la eduación en seguridad y la
  responsabilidad de revisar el cumplimiento de los requisitos.
\item[Evaluación del impacto] Identificación de posibles impactos que
  puede ocasionar los riesgos, ya sea de manera tangible (financión) o
  intagible (reputación). Se realizan listas de prioridades,
  evaluaciones y métricas de impactos. Incluye la identificación y
  monitorización de los mismos.
\item[Evaluación de los riesgos] Identifica, analiza y evalúa los
  riesgos en seguridad en los que se pueda ver envualta la
  organización. Se selecciona un método de evaluación, evalúa el nivel
  de exposición y su certeza\footnote{Métrica de que realmente
    ocurra.}. Incluye también la priorización de riesgos y su
  monitorización. 
\item[Evaluación de las amenazas] Identifica las amenazas en la
  seguridad, sus propiedades y caracteristicas. Incluye la
  identificación de amenazas naturales, humanas, métricas, capacidad
  de respuesta y la probabilidad de suceder. También la monitorización
  y sus características.
\item[Evaluación de vulnerabilidades] Se identifican las
  vulnerabilidades a las que se enfrente la organzación, eligiendo
  unos métodos, técnicas y criterios para caracterizarlas. Recolecta
  la información relacionada con las vulnerabilidades y monitoriza sus
  cambios.
\item[Construcción de un argumento de aseguración] Proceso para
  convencer al cliente de la seguridad propia. Se identifican los 
  objetivos, las medidas de control y monitorización. Recoge
  evidencias y realiza análisis sobre ellas. Con ello se contruye el
  argumento acerca de la propia seguridad.
\item[Coordinación de la seguridad] Se asegura de que todas las partes
  son conscientes y están involucradas en las actividades de
  ingeniería de la seguridad. Define la coordinación, sus mecanismos y
  facilitación de los mismos.
\item[Monitorización de la postura en la seguridad] Se asegura de que
  todos los fallos son identificados y comunicados.
\item[Proporcionar información en seguridad] Proporciona a arquitectos,
  diseñadores, implementadores y/o usuarios la información que
  necesitan.
\item[Especificación de las necesidades de seguridad] Indentifica y
  relaciona las necesidades acerca del sistema. Define las bases en
  seguridad para satisfacer los requisitos legales, políticos y
  organizativos.
\item[Verificación y validación de la seguridad] Se asegura del
  cumplimiento de los compromisos adquiridos.
\end{description}

La \emph{SSE-CMM®} define además otros procesos organizativos en
anexos, continuación de los aquí descritos.

\end{document}
