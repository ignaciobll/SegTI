%% \documentclass[11pt,twocolumn]{article}
\documentclass[11pt,a4paper]{article}

\usepackage[utf8]{inputenc}
\usepackage[spanish]{babel}

\usepackage{blindtext}
\usepackage{scrextend}
\usepackage{enumitem}

\usepackage{graphicx}
\usepackage{pgfplots}
\usepackage{tikz}
\usepackage{tikzscale}

\usepackage{subfigure}

\usepackage{caption}

\title{Análisis del efecto Avalancha sobre RipeMD320}

\author{Ignacio Ballesteros González}

\date{\today}

\begin{document}
\maketitle
\tableofcontents
\begin{abstract}
  Para la realización de la práctica primera de la asignatura de
  \emph{Seguridad de las Tecnologías de la Información} se ha
  realizado el análisis del efecto \emph{avalancha} sobre el algoritmo de
  \emph{hash} \emph{RipeMD320}.
\end{abstract}

\section{Introducción}
El algoritmo de \emph{hash RipeMD} se desarrolló en la universidad
\emph{Katholieke Universiteit Leuven} y se publicó su primera versión
en 1996. El algoritmo se mejoró en siguientes versiones (\emph{160,
  256}) para evitar colisiones hasta llegar a la versión
\emph{RipeMD320} que será la que aquí se analice.

\subsection{Herramientas utilizadas}

Para la posible replicación de este análisis, se detallan las
herramientas utilizadas en el desarrollo del mismo:

\begin{description}[align=left] %, noitemsep]
\item  [Java 1.8] Plataforma de desarrollo.
\item  [Eclipse] Entorno de desarrollo para programación en
  \emph{Java}. \emph{Neon 3 Release (4.6.3)}
\item [Bouncy Castle] Librería con funciones
  criptográficas, entre ellas incluida la necesaria para
  \emph{RipeMD320}.
\item [Apache Commons Math] Librería \emph{Java} para cálculos estadísticos.
\item [\LaTeX] Realización de figuras y de este mismo
  documento.
\end{description}

\section{Estructura del análisis}

Para la realización del análisis se ha seguido el siguiente
procedimiento:

\begin{enumerate}
\item Comenzamos el análisis con una cadena (\textbf{A}) \emph{pseudo-aleatoria
  uniformemente distribuida} de 320 \emph{bits}.
\item Calculamos el \emph{hash RipeMD320} sobre la cadena \textbf{A}.
\item Variamo un \emph{bit} aleatorio de la cadena \textbf{A}. Esta nueva
  cadena será la cadena \textbf{B}.
\item Calculamos el \emph{hash RipeMD320} sobre la cadena \textbf{B}.
\item Calculamos la \emph{distancia de Hamming} entre el \emph{hash}
  de la cadena \textbf{A} y \textbf{B}.
\item Generamos una nueva cadena origen y repetimos estos pasos
  \emph{n} veces, dependiendo del tamaño de la muestra que queramos
  analizar.
\item Aplicamos un análisis sobre las \emph{distancias de Hamming}
  obtenidas.
\end{enumerate}

\section{Datos recogidos}

Los análisis se ha realizado según el tamaño de la
muestra. Los datos estadísticos asociados se presentan en la siguiente
tabla:

\begin{center}
  \vspace{-15pt}
  \label{tab:medidas}
  \begin{tabular}[c]{| r | *{6}{c} |}
  \hline
  Población  & Media  & Moda & Mediana & Desviación & Asimetría & Curtosis \\ \hline
  10         & 158.36 & 157  & 158     & 7.43       & -0.82     & 3.05     \\
  100        & 159.93 & 165  & 161     & 8.68       & 0.29      & 0.85     \\
  1000       & 160.64 & 165  & 161     & 9.10       & -0.11     & -0.23    \\
  10000      & 160.29 & 161  & 160     & 8.99       & 0.02      & 0.04     \\
  100000     & 160.23 & 161  & 160     & 8.97       & 0.00      & 0.00     \\
  1000000    & 160.24 & 161  & 160     & 8.97       & 0.00      & -0.01    \\
  100000000  & 160.03 & 160  & 160     & 8.97       & 0.00      & 0.00     \\
  \hline
  \end{tabular}
  \captionof{table}{Tabla de medidas estadísticas}

\vspace{-10pt}
\end{center}

En orden creciente según la población, visualización de los datos en
un histograma. En el eje $y$ se representan el número de ocurrencias
de la \emph{distancia de Hamming} dada (eje $x$).

%% Multi plot 2x2
\begin{figure}[h]
  \centering
  \vspace{-10pt}
  \subfigure[Población de $10$]{\includegraphics[width=0.375\textwidth]{./figures/histograma10.tikz}}
  \subfigure[Población de $100$]{\includegraphics[width=0.375\textwidth]{./figures/histograma100.tikz}}
  \\
  \subfigure[Población de $1000$]{\includegraphics[width=0.375\textwidth]{./figures/histograma1000.tikz}}
  \subfigure[Población de $10000$]{\includegraphics[width=0.375\textwidth]{./figures/histograma10000.tikz}}
  \vspace{-10pt}
  \caption{Histogramas de población pequeña}
  \vspace{-10pt}
  \label{fig:small}
\end{figure}

\begin{figure}
  \centering
  \subfigure[Población de $100000$ ]{\includegraphics[width=0.45\textwidth]{./figures/histograma100000.tikz}}
  \subfigure[Población de $1000000$]{\includegraphics[width=0.4\textwidth]{./figures/histograma1000000.tikz}}
  \caption{Muestras de población mediana}
  \label{fig:med}
\end{figure}

\begin{figure}
  \centering
  \vspace{-30pt}
  \includegraphics[width=\linewidth]{./figures/histograma.tikz}
  \caption{Muestra de gran tamaño}
  \vspace{-10pt}
  \label{fig:big}
\end{figure}

\section{Valoración de los resultados}

Cuando se trata de comprobar el efecto \emph{avalancha} sobre una
función \emph{hash} se espera que cuando se cambie una pequeña parte
de la cadena de entrada (por ejemplo, un \emph{bit}), aproximadamente
la mitad de los \emph{bits} de la cadena de salida cambien también.

Que solo cambie aproximadamente la mitad de los bits quiere decir que
la \emph{distancia de Hamming} asociada a la modificación de un solo
\emph{bit} sea de aproximadamente la \textbf{mitad} de la cadena. En
en el caso de la función \emph{RipeMD320} las cadenas son de \emph{320
  bits}, por lo que la \emph{distancia de Hamming} asociada en un
cambio de \emph{bit} se concentraría cerca de $160$.

Como podemos comprobar según los datos estadísticos del \emph{Cuadro 1}, la
media, mediana y moda tienden a $160$. Por otro lado, cuanto más
simétrica sea la distribución, más \emph{uniformemente distribuida} será y
más se aproximará al comportamiento de un \emph{Oráculo Aleatorio}.

En base a los datos obtenidos de simetría cercana a $0$ (a falta de
cifras significativas), una curtosis que indica una fuerte proximidad
a la mediana, la cercanía entre \emph{mediana}, \emph{media} y
\emph{moda}; \textbf{se determina que la función de \emph{hash
    RipeMD320} exhibe un fuerte \emph{efecto avalancha}}.

\end{document}
